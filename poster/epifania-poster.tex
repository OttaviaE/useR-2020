\documentclass[a0,landscape]{a0poster}
\usepackage{hyperref}
\usepackage{multicol} % This is so we can have multiple columns of text side-by-side
\columnsep=80pt % This is the amount of white space between the columns in the poster
%\columnseprule=2pt % This is the thickness of the black line between the columns in the poster

\usepackage[svgnames]{xcolor} % Specify colors by their 'svgnames', for a full list of all colors available see here: http://www.latextemplates.com/svgnames-colors

\usepackage{times} % Use the times font
%\usepackage{palatino} % Uncomment to use the Palatino font

\usepackage{graphicx} % Required for including images
%\graphicspath{{C:/Users/huawei/Desktop/images/}} % Location of the graphics files
\usepackage{booktabs} % Top and bottom rules for table
\usepackage[font=small,labelfont=bf]{caption} % Required for specifying captions to tables and figures
\usepackage{amsfonts, amsmath, amsthm, amssymb} % For math fonts, symbols and environments
\usepackage{wrapfig} % Allows wrapping text around tables and figures
\usepackage[english]{babel}
\usepackage[scale=2]{ccicons}
\usepackage{picture}
\usepackage{tikz}
\usepackage[absolute,overlay]{textpos}
\usepackage{spot}
\usepackage{tabularx}
\usepackage{multirow}
\usepackage{caption}
\usepackage{subcaption}
\usepackage{spot}
\usepackage{floatrow}
\usepackage{setspace}
\usepackage{listings}
\usepackage[english]{babel}


\usetikzlibrary{calc, positioning, shapes.arrows}
\definecolor{title}{RGB}{20,68,199}
\definecolor{comp}{RGB}{35, 99, 70}
\definecolor{inc}{RGB}{191, 13, 43}
\definecolor{oldlavender}{rgb}{0.47, 0.41, 0.47}
\definecolor{command}{rgb}{153,112,171}
\definecolor{comment}{rgb}{90,174,97}


\lstset{language=R,
	basicstyle=\small\ttfamily,
	stringstyle=\color{blue},
	%otherkeywords={0,1,2,3,4,5,6,7,8,9},
	morekeywords={TRUE,FALSE, clean\_iat, d\_plot, d\_distr, non, clean, iat, mappingA, mappingB, sbj, id, block, mapA, mapB, test, practice, latency, id, trial, eliminate, accuracy, iat\_data, clean\_sciat, dscore, Dsciat, 
	sciat, d, sciat1, sciat2, multiple, scores,1,2, multi, ds, graph},	
	deletekeywords={<, -, data, \_, order, plot, col},
	keywordstyle=\color{purple}\bfseries,
	commentstyle=\color{green!40!blue},
}

\begin{document}

%----------------------------------------------------------------------------------------
%	POSTER HEADER 
%----------------------------------------------------------------------------------------

% The header is divided into two boxes:
% The first is 75% wide and houses the title, subtitle, names, university/organization and contact information
% The second is 25% wide and houses a logo for your university/organization or a photo of you
% The widths of these boxes can be easily edited to accommodate your content as you see fit

\begin{minipage}[b]{\linewidth}
	\begin{textblock*}{12cm}(10cm,0.5cm)
		\includegraphics[width=\linewidth]{implicitMeasures.png}\\
	\end{textblock*}
	
	\begin{center}
\veryHuge \textcolor{title}{\textbf{Scoring the implicit:}}
\veryHuge\textcolor{title}{\textbf{The \texttt{implicitMeasures} package}}\\[2cm] % Subtitle	

\LARGE \textbf{Ottavia M. Epifania, Pasquale Anselmi, \& Egidio Robusto}\\[0.5cm] % Author(s)
\Large University of Padova\\[0.4cm] % University/organization
\Large \texttt{marinaottavia.epifania@phd.unipd.it}\\	
	\end{center}

	\begin{textblock*}{12cm}(98cm,0.5cm)
		\begin{tabular}{l}
			\href{https://github.com/OttaviaE/implicitMeasures}{\includegraphics[width=\linewidth]{github.png}}
		\end{tabular}
	
\end{textblock*}

\end{minipage}
%
%\begin{minipage}[b]{0.25\linewidth}
%\includegraphics[width=3cm]{logo.png}\\
%\end{minipage}

%\vspace{0.5cm} % A bit of extra whitespace between the header and poster content

%----------------------------------------------------------------------------------------
\setlength{\columnseprule}{0.4pt}
 % This is how many columns your poster will be broken into, a portrait poster is generally split into 2 columns

\vspace{5mm}

\begin{multicols*}{3}
%----------------------------------------------------------------------------------------
%	ABSTRACT
%----------------------------------------------------------------------------------------

\begin{center}
	\huge \textbf{\textcolor{title}{Introduction}}
\end{center}

\vspace{5mm}
\large

Social sciences are showing a growing interest for the implicit investigation of attitudes. The Implicit Association Test (IAT) \cite{Greenwald1998} and the Single Category IAT (SC-IAT) \cite{karpinski2006} are the mostly common employed measures for this aim \cite{review}.
Both tests result in a differential score (the so-called \emph{D-score}) expressing respondents’ bias in performing the categorization task between conditions. There is a single scoring for the SC-IAT \cite{karpinski2006} and six algorithms for scoring the IAT \cite{Greenwald2003}.

\texttt{implicitMeasures} is an \texttt{R} package for scoring both the IAT and the SC-IAT. The package basic functioning is illustrated. 

\vspace{5mm}
\begin{center}
	\Large \textbf{\textcolor{title}{The Implicit Association Test}}
\end{center}
\vspace{5mm}

The IAT is usually composed of seven blocks:

\vspace{0.4cm}
\begin{center}
	\begin{tabular}{llll}
		\multicolumn{4}{c}{IAT structure \label{tab:iat}}\\
		\toprule
		Block & Function & Left response key & Right response key  \\
		\midrule
		\emph{B1} & Practice & Object 1 & Object 2 \\
		\emph{B2} & Practice & \emph{Good} & \emph{Bad} \\
		\textcolor{comp}{\emph{B3}} & \textcolor{comp}{Associative Practice Mapping A} & \textcolor{comp}{Object 1 + \emph{Good}} & \textcolor{comp}{Object 2 + \emph{Bad}} \\
		\textcolor{comp}{\emph{B4} } & \textcolor{comp}{Associative Test Mapping A} & \textcolor{comp}{Object 1 + \emph{Good}} & \textcolor{comp}{Object 2 + \emph{Bad}} \\
		\emph{B5} & Practice & Object 2 & Object 1 \\
		\textcolor{inc}{\emph{B6}} & \textcolor{inc}{Associative Practice Mapping B} & \textcolor{inc}{Object 2 + \emph{Good}} & \textcolor{inc}{Object 1 + \emph{Bad}} \\
		\textcolor{inc}{\emph{B7}} & \textcolor{inc}{Associative Test Mapping B} & \textcolor{inc}{Object 2 + \emph{Good}} & \textcolor{inc}{Object 1 + \emph{Bad} }\\
		\bottomrule
	\end{tabular}
\end{center}

\vspace{5mm}

The computation of the IAT \emph{D-score} is rather easy but 6 algorithms are available:

\vspace{5mm}
\begin{tabular}{lll}
	\large	
	$\text{\emph{D-score}} = \frac{M_{\text{MappingA}} - M_{\text{MappingB}}}{sd_{\text{pooled}}}$
	    & &	\begin{tabular}{lll}
		\multicolumn{3}{c}{IAT \emph{D-score} algorithms}\\
		\toprule
		\emph{D-score} & Error Replacement & Lower tail treatment \\
		\midrule
		\emph{D1} & Built-in & No \\ 
		\emph{D2} & Built-in & $<$ 400 ms \\ 
		\emph{D3} & Mean + 2 \emph{sd} & No \\ 
		\emph{D4} & Mean + 600 ms & No \\ 
		\emph{D5} & Mean +2 \emph{sd} & $<$ 400 ms \\ 
		\emph{D6} & Mean + 600 ms & $<$ 400 ms \\ 
		\bottomrule
		\multicolumn{3}{l}{\emph{Note:} Trials with rt $>$ 10,000 ms are discarded}\\
	\end{tabular}\\
\end{tabular}

\vspace{5mm}
\begin{center}
	\Large \textbf{\textcolor{title}{The Single Category Implicit Association Test}}
\end{center}
\vspace{5mm}

Same assumption, same score computation as the IAT \textbf{BUT} just one target object: 
\vspace{5mm}

\begin{center}
	\begin{tabular}{llll}
		\multicolumn{4}{c}{SC-IAT structure \label{tab:sciat}}\\
		\toprule
		Block & Function & Left response key & Right response key  \\
		\midrule
		\emph{B1} & Associative Practice Mapping A & Object 1 + \emph{Good} & \emph{Bad} \\
		\textcolor{comp}{\emph{B2} } & \textcolor{comp}{Associative Test Mapping A} & \textcolor{comp}{Object 1 + \emph{Good}} & \textcolor{comp}{\emph{Bad}} \\
		\emph{B3} & Associative Practice Mapping B & \emph{Good} & Object 1 + \emph{Bad} \\
		\textcolor{inc}{\emph{B4}} & \textcolor{inc}{Associative Test Mapping B} & \textcolor{inc}{ \emph{Good}} & \textcolor{inc}{Object 1 + \emph{Bad} }\\
		\bottomrule
	\end{tabular}
\end{center}

\vspace{5mm}
The SC-IAT \emph{D-score} is computed as the IAT \emph{D-score} by considering just the responses in Blocks \emph{B2} and \emph{B4}. Error responses are replaced with the average response time added with a penalty of 400 ms. Responses faster than 350 ms are discarded. The administration might include a Response Time Window (rtw) at 1,500 ms after which the stimulus disappears and the response is considered a non-response. 



\vfill\null
\columnbreak

\begin{center}
	\huge \textbf{\textcolor{title}{IAT example}}
\end{center}

The examples are based on the toy data set in \texttt{implicitMeasures} package. 
\vspace{3mm}
\begin{lstlisting}
data("raw_data") # load toy data
iat <- clean_iat(raw_data, # raw data set 
           sbj_id = "Participant", #Respondents' IDs
           block_id = "blockcode", #Blocks labels
           mapA_practice = "practice.iat.Milkbad", #Label B3
           mapA_test = "test.iat.Milkbad", # Label B4
           mapB_practice = "practice.iat.Milkgood", #Label B6
           mapB_test = "test.iat.Milkgood", #Label B7
           latency_id = "latency", #Response times
           accuracy_id = "correct", #Accuracy responses)
\end{lstlisting}
\vspace{3mm}

Function \textbf{\textcolor{purple}{\texttt{clean\_iat()}}} results in a list, in which the first object is a data frame with class \textcolor{purple}{\texttt{iat\_clean}}. This data set can be passed to function \textbf{\textcolor{purple}{\texttt{computeD()}}}:
 \vspace{3mm}
\begin{lstlisting}
iat_dscore <- computeD(iat[[1]], #iat_clean object
                       Dscore =  "d2") #D-score algorithm
\end{lstlisting}
\vspace{3mm}

 \textbf{\textcolor{purple}{\texttt{iat\_dscore}}} is an object of class \textcolor{purple}{\texttt{dscore}}. This object can be passed to functions \textbf{\textcolor{purple}{\texttt{descript\_d()}}} (i.e., print a table of descirptive statistics of the results, even in \LaTeX), \textbf{\textcolor{purple}{\texttt{IATrel()}}} (i.e., compute teh IAT reliability), and to functions for obtaining graphical representation of the results, either at the individual respondente with function \textbf{\textcolor{purple}{\texttt{d\_plot()}}} or at the sample level with function \textbf{\textcolor{purple}{\texttt{d\_distr()}}}: 
\vspace{3mm}
		\begin{center}
			\begin{tabular}{c c}
				\includegraphics[width=0.40\linewidth]{dplot.pdf}
				&
				\includegraphics[width=0.40\linewidth]{ddistr.pdf} \\
				\textbf{\textcolor{purple}{\texttt{d\_plot()}}} & \textbf{\textcolor{purple}{\texttt{d\_distr()}}}\\
			\end{tabular}
		\end{center}
\vspace{3mm}

The object with class \textcolor{purple}{\texttt{iat\_clean}} can also be passed to function \textbf{\textcolor{purple}{\texttt{multi\_dscore()}}}, which simultaneously compute multiple \emph{D-score} algorithms: 
\vspace{3mm}
\begin{lstlisting}
multiple_scores <- multi_dscore(iat[[1]], 
	            ds = "error-inflation") #Which algorithm?
\end{lstlisting}
\vspace{3mm}

Function \textbf{\textcolor{purple}{\texttt{multiple\_scores()}}} results in a list containing a data set with a number of columns equal to the number of computed alrgorithms plus a column for respondnents' IDs and a graph with the distribution resulting from the different algorithms: 
\vspace{3mm}
	\begin{tabular}{l}
		\begin{lstlisting}
		multiple_scores$graph # Only select the graphical output
		\end{lstlisting}\\
		\\
		\multicolumn{1}{c}{ \includegraphics[width=0.5\linewidth]{multid.pdf}}  \\
	\end{tabular}


 
\vfill\null
\columnbreak

\begin{center}
	\huge \textbf{\textcolor{title}{SC-IAT example}}
\end{center}

The \texttt{raw\_data} toy data set contains data from two distinct data SC-IATs. Data from both SC-IATs can be simulatenously clenaed and prepared for the computation by using \textbf{\textcolor{purple}{\texttt{clean\_sciat()}}} function: 
\vspace{3mm}
\begin{lstlisting}
data("raw_data") # load toy data 
sciat <- clean_sciat(raw_data, # raw data set
                      sbj_id = "Participant", #Column with respondents' IDs
                      block_id = "blockcode", #Column with blocks labels
                      latency_id = "latency", #Column with latency responses
                      accuracy_id = "correct", #Column with accuracy responses
                      block_sciat_1 = c("test.sc_dark.Darkbad", #1st SC-IAT block labels
                      "test.sc_dark.Darkgood"),
                      block_sciat_2 = c("test.sc_milk.Milkbad", #2nd SC-IAT block labels
                      "test.sc_milk.Milkgood"),
                      trial_id  = "trialcode", #Column with trial labels
                      trial_eliminate = c("reminder", 
                      "reminder1")) #Trials to eliminate
\end{lstlisting}	
\vspace{3mm}

 \textbf{\textcolor{purple}{\texttt{clean\_sciat()}}} results in a list containing two data sets of class \textcolor{purple}{\texttt{sciat\_clean}}, one for each SC-IAT. These data sets can be individually passed to \textbf{\textcolor{purple}{\texttt{Dsciat()}}} to compute the SC-IAT \emph{D-score}: 
\vspace{3mm}
\begin{lstlisting}
 d_sciat1 <- Dsciat(sciat[[1]], # 1st SC-IAT data
                   mappingA = "test.sc_dark.Darkbad", # Mapping A Label
                   mappingB = "test.sc_dark.Darkgood", # Mapping B Label
                  non_response = "alert") # Label for responses over rtw
 d_sciat2 <- Dsciat(sciat[[2]], # 2nd SC-IAT data
                   mappingA = "test.sc_milk.Milkbad", # Mapping A Label
                   mappingB = "test.sc_milk.Milkgood", # Mapping B Label
                   non_response = "alert") # Label for responses over rtw
\end{lstlisting}	
\vspace{3mm}

\textbf{\textcolor{purple}{\texttt{Dsciat()}}} results in objects of class \textcolor{purple}{\texttt{dsciat}}, which can be passed to functions \textbf{\textcolor{purple}{\texttt{descript\_d()}}}, \textbf{\textcolor{purple}{\texttt{d\_plot()}}}, and \textbf{\textcolor{purple}{\texttt{d\_distr()}}}. 

Results obtained from multiple SC-IATs can be plotted together by using \textbf{\textcolor{purple}{\texttt{multi\_dsciat()}}}: 
\vspace{3mm}
		\begin{center}
		\begin{tabular}{c c}
			\includegraphics[width=0.40\linewidth]{sciatDefault.pdf}
			&
			\includegraphics[width=0.40\linewidth]{multiSciatPoints.pdf} \\
			\textbf{\textcolor{purple}{\texttt{multi\_dsciat()}}}  (Default) & \textbf{\textcolor{purple}{\texttt{multi\_dsciat()}}} (Customized) \\
		\end{tabular}
		\end{center}


\vspace{5mm}

\textcolor{title}{Last but not least:} All graphical functions are based on \texttt{ggplot2} \cite{ggplot2} and can be further modified by the useRs! 

\vspace{5mm}
---------------------------------------
\footnotesize
%\nocite{*} % Print all references regardless of whether they were cited in the poster or not
\bibliographystyle{plain} % Plain referencing style
\bibliography{poster} % Use the example bibliography file sample.bib

%----------------------------------------------------------------------------------------

\end{multicols*}
\end{document}